\documentclass[12pt,a4paper]{article}

\PassOptionsToPackage{hyphens}{url} % \sloppy\url{}
\usepackage[colorlinks=true,linkcolor=black,urlcolor=blue,citecolor=blue]{hyperref} % \href{https://www.youtube.com/}{Youtube} or \url{}

\usepackage{multicol}

\parindent =0cm % No indentation

\title{Solving the time-independent Schrödinger equation for a hindered methyl rotor potential with \textit{QRotor.py}}
\author{Pablo Gila-Herranz}
\date{2024-03-27}

\begin{document}
\maketitle


\section*{Time-independent Schrödinger equation for a hindered rotor}

The time-independent Schrödinger equation is
$$
H\Psi(\varphi)=E\Psi(\varphi)
$$

The hamiltonian for a hindered methyl rotor can be expressed as a sum of the kinetic rotational energy and the potential energy,
$$
H = -B \frac{d^2\Psi}{d\varphi^2} - V(\varphi)
$$

with
$$
B = \frac{1}{2I}=\frac{1}{2\sum_{i}m_{i}r_{i}^{2}}
$$

The potential can be adjusted to the following form, where the coefficients are calculated via electronic calculation methods \cite{titov2023},
$$
V(\varphi)=c_{0}+c_{1}\sin(3\varphi)+c_{2}\cos(3\varphi)+c_{3}\sin(6\varphi)+c_{4}\cos(6\varphi)
$$


\section*{Finite Difference Method}

The time-independent Schrödinger equation is a second-order differential equation. It can be solved numerically using the finite difference method. The first derivative can be approximated as
$$
\frac{d\Psi}{d\varphi} = \frac{\Psi(\varphi+\Delta\varphi)-\Psi(\varphi)}{\Delta\varphi}
$$


\bibliographystyle{IEEEtran} % Estilo
\bibliography{QRotor_refs} % documento .bib

\end{document}

